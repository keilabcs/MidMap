\documentclass{beamer}

\usepackage[utf8]{inputenc}
\usepackage[T1]{fontenc}
\usepackage[spanish,english,portuguese]{babel}

\usetheme{Boadilla}
\usecolortheme{beaver}

%%%%%%%%%% CRONOGRAMA DE ATIVIDADE %%%%%%%%%%%%%%%
\usepackage{pgfgantt} % Fazer Cronograma Gantt	
\usepackage{graphicx} 
%%%%%%%%%%%%%%%%%5%%%%%%%%%%%%%%%5%%%%%%%%%%%%%%%%	

\usepackage{natbib}
\usepackage{placeins} % Forçar Tabela/Figura fica abaixo do texto

%%%%%% MAPA MENTAL %%%%%%%%
\usepackage{tikz,times}
\usetikzlibrary{positioning,shadows.blur}
\usepackage{verbatim}
\usepackage{geometry}
\usetikzlibrary{mindmap,trees,backgrounds,shadows}


\definecolor{color_mate}{RGB}{255,255,128}
\definecolor{color_plas}{RGB}{255,128,255}
\definecolor{color_text}{RGB}{128,255,255}
\definecolor{color_petr}{RGB}{255,192,192}
\definecolor{color_made}{RGB}{192,255,192}
\definecolor{color_meta}{RGB}{192,192,255}
%%%%5%%%%%%%%%%%%%%%%%%%%%%5%

%%5%%%%%%%%%%%%%%%%%%%%%%%%%%%5%%%%%
\title{Seminários em Modelagem Computacional de Conhecimento}
\author{Keila Barbosa Costa dos Santos}
\institute{Programa de Pós-Graduação em Modelagem Computacional do Conhecimento \\
  Universidade Federal de Alagoas}

\begin{document}

%%5%%%%%%%%%%%%%%%%%%%%%%%%%%%5%%%%% (1)
\begin{frame}
  \titlepage
\end{frame}
%%5%%%%%%%%%%%%%%%%%%%%%%%%%%%5%%%%%

%%5%%%%%%%%%%%%%%%%%%%%%%%%%%%5%%%%% (2)
\begin{frame}

\begin{block}{}
Linha de Pesquisa: Modelos Quantitativos e de Simulação.
\end{block}

\begin{block}{}
Título: Análise Bibliométrica de Currículos Lattes.
\end{block}

\begin{block}{}
Orientador: Alejandro C. Frery
\end{block}

\end{frame}
%%5%%%%%%%%%%%%%%%%%%%%%%%%%%%5%%%%%

%%5%%%%%%%%%%%%%%%%%%%%%%%%%%%5%%%%% (3)
\begin{frame}{Definição de Bibliometria}

\begin{block}{}
A bibliometria é um campo das áreas da biblioteconomia e da ciência da informação que aplica métodos estatísticos e matemáticos para analisar e construir indicadores sobre a dinâmica e evolução da informação científica e tecnológica de determinadas disciplinas, áreas, organizações ou países.
\end{block}

\begin{block}{}
Bibliometria é aplicação da matemática e métodos estatísticos para livros e outras mídias de comunicação~\footnote{Pritchard, 1969, p 349.}.
\end{block}

\end{frame}
%%5%%%%%%%%%%%%%%%%%%%%%%%%%%%5%%%%%

%%5%%%%%%%%%%%%%%%%%%%%%%%%%%%5%%%%% (4)
\begin{frame}{Mapa Mental da Área (tikz - mindmap)}

\begin{figure}[!h]
\begin{center}
\scalebox{0.45}{ % PS
\begin{tikzpicture}[every annotation/.style = {draw,fill = white, font = \Large}]
\path[mindmap,concept color=black!40,text=white,every node/.style={concept,circular drop shadow},
root/.style    = {concept color=black!40,font=\large\bfseries,text width=10em},
level 1 concept/.append style={font=\Large\bfseries,sibling angle=90,text width=7.7em,
level distance=15em,inner sep=0pt},level 2 concept/.append style={font=\bfseries,level distance=9em},
]
node[root] {Biblioteconomia \& ciência da Informação} [clockwise from=0]
      child[concept color=blue!60] {
 node {Webometria}[clockwise from=90]
     }
      child[concept color=blue] {
 node[concept]{Bibliometria}[clockwise from=-30]
        child {node[concept color=yellow!60!black] {Prod. Científica}[clockwise from=40]
          child {node[concept color=yellow!60!black] {Área}[clockwise from=90]
            child {node[concept color=yellow!60!black] {Com- putação}[clockwise from=30]
              child {node[concept color=yellow!60!black] {Gênero}}
             }
             }
            child {node[concept] {Disciplinas}}
            child {node[concept] {Países}}
            child {node[concept] {Organi- zações}}
           }
        child { node[concept] {Prod. de Periódicos}}
        child { node[concept] {Freq. de Palavras}}
      }
      child[concept color=green!40!black] {
 node[concept]{Informetria}[clockwise from=310]
      }
      child[concept color=red] {
 node[concept]{Cientometria}[clockwise from=270]
      };
\end{tikzpicture}
} % PS
\end{center}
\end{figure}
\end{frame}
%%5%%%%%%%%%%%%%%%%%%%%%%%%%%%5%%%%%

%%%%%%%%%%%%%%%%%%%%%%%%%%%%%%%%%%%% (5)
\begin{frame}{Objetivo}

\begin{block}{}
Medir com base no Lattes o impacto da produção cientifica nos últimos dez anos em função do gênero dos autores na área de Ciência da Computação.
\end{block}

\end{frame}
%%%%%%%%%%%%%%%%%%%%%%%%%%%%%%%%%%%%

%%%%%%%%%%%%%%%%%%%%%%%%%%%%%%%%%%%% (6)
\begin{frame} {Metodologias}

\begin{block}{Métricas de Produção Bibliográfica}
\begin{itemize}
\item \textit{\textcolor{red}{Artigos completos publicados em periódicos}};
\item Livros publicados/organizados ou edições;
\item Capítulos de livros publicados;
\item Textos em jornais de notícias/revistas;
\item \textit{\textcolor{red}{Trabalhos completos publicados em anais de congressos}};
\item Resumos expandidos publicados em anais de congressos;
\item Resumos publicados em anais de congressos;
\item Artigos aceitos para publicação;
\item Apresentações de trabalho. 
\end{itemize}
\end{block}

\end{frame}
%%%%%%%%%%%%%%%%%%%%%%%%%%%%%%%%%%%%

%%%%%%%%%%%%%%%%%%%%%%%%%%%%%%%%%%%% (7)
\begin{frame} {Metodologias}

\begin{block}{Dados}
\begin{itemize}
\item Plataforma Lattes;
\item Página das Instituições;
\item Web of Science;
\item Scopus.
\end{itemize}
\end{block}

\end{frame}

%%%%%%%%%%%%%%%%%%%%%%%%%%%%%%%%%%%%

%%%%%%%%%%%%%%%%%%%%%%%%%%%%%%%%%%%% (8)
\begin{frame} {Ferramentas}

\begin{block}{}
\begin{itemize}
\item scripLattes - Extração dos dados.
\end{itemize}
\end{block}

\begin{block}{}
\begin{itemize}
\item Gephi - Análise da Rede.
\end{itemize}
\end{block}

\begin{block}{}
\begin{itemize}
\item R - Análise dos Dados.
\end{itemize}
\end{block}

\end{frame}
%%%%%%%%%%%%%%%%%%%%%%%%%%%%%%%%%%%%

%%5%%%%%%%%%%%%%%%%%%%%%%%%%%%5%%%%% (9)
% MAPA MENTAL %
\begin{frame}{Mapa Mental Conceitual (tikz - mindmap)}
\begin{figure}[!h]
\begin{center}
\scalebox{0.36}{ % PS
\begin{tikzpicture}[mindmap,level 1 concept/.append style={level distance=200,sibling angle=60},extra concept/.append style={text=black}]

% IMPACTO
\path[mindmap,concept color=color_mate,text=black] node[concept] {Impacto da Produção Científica na Ciência da computação em Função do Gênero}[clockwise from=30]
  
    % FATORES DETERMINANTES
    child[concept color= color_petr] {
      node[concept, inner sep = 0mm ] {Fatores Determinantes dos Padrões Observados}  [clockwise from=150]
      child { node[concept] {Cognitivo}}
      child { node[concept] {Social}}
      child { node[concept] {Histórico}}
      child { node[concept] {Geopolítico}}
      child { node[concept] {Econômico}}
    }
    %
    % FATORES AVALIADOS
    child[concept color=color_text] {
      node[concept] {Indicadores Bibliométricos} [clockwise from=60]
      child { node[concept] {Qualidade Científica}}
      child { node[concept] {Atividade Científica}}
      child { node[concept] {Impacto Científico}}
      child { node[concept] {Associações Temáticas}}
      }
    %
    % ESTADO DA ARTE
    child[concept color=color_meta] {
      node[concept] {Estado da Arte}  [clockwise from=30]
          child { node[concept] {Palavras Chaves}}
          child { node[concept] {Revisão Cronológica}}
          child { node[concept] {Revisão Conceitual}}
    }
    %
    % CNPQ
    child[concept color=color_plas] {
      node[concept] {CNPq - Plataforma Lattes} [clockwise from=-120]
      child { node[concept] {Coleta de Dados}} [clockwise from=0]
          child { node[concept] {Análise de dados}
          child [concept color=green!40]{ node{R}}}
          child { node[concept] {Ferramentas}
          child [concept color=green!40]  { node {ScriptLattes}}
          child [concept color=green!40]  { node {LattesX}}
          }
          child { node[concept] {Tratamento de dados}}
          }
    %
    % ISI - JCR
    child[concept color=color_made] {
      node[concept] {Ferramentas Bibliométricas} [clockwise from=-160]
          child { node[concept] (A) {Scopus}}
          child { node[concept]     {Web of Science (WoS)}[clockwise from=130]
          child { node[concept] (D) {JCR}}
          }
          child { node[concept] (C) {Google Scholar Metrics}}
          }
    %
    % HIPOTESES
    child[concept color=orange] {
      node[concept] {Hipóteses} [clockwise from=150]
          child { node[concept] {Hº = H e M = Impacto}}
          child { node[concept] {H1 = Homens Tem Maior Impacto}}
          child { node[concept] {H2 = Mulheres Tem Maior Impacto}}
       };
    
\begin{pgfonlayer}{background}
  \draw [left color=blue, right color=green!50!black, draw=white, decorate,decoration=circle connection bar] (A) -- (D);
  \draw [left color=blue, right color=green!50!black, draw=white, decorate,decoration=circle connection bar] (C) -- (D); 
\end{pgfonlayer}
   
\end{tikzpicture}
} % PS
\end{center}
\end{figure}	
\end{frame}
%%5%%%%%%%%%%%%%%%%%%%%%%%%%%%5%%%%%

%%5%%%%%%%%%%%%%%%%%%%%%%%%%%%5%%%%% (10)
\begin{frame}{Cronograma  de Atividade Diagrama Gantt (pgfgantt)}

\begin{figure}[!h]	
\scalebox{0.58}{ % PS
\begin{ganttchart}[y unit title=0.8cm, y unit chart=0.6cm, vgrid,hgrid, title height=0.5, bar/.style={draw,fill=magenta}, bar incomplete/.append style={fill=yellow!50}, bar height=0.75]{1}{24}

 \gantttitle{1 Ano}{12}
 \gantttitle{2 Ano}{12} \\
 \gantttitlelist{1,...,12}{1}
 \gantttitlelist{13,...,24}{1}\\
 
 \ganttbar[progress=70]{Disciplinas Obrigatórias}{1}{06} \\
 
\ganttgroup{Estado da Arte}{1}{21}\\ 
 
\ganttgroup{Planejamento da Observação}{07}{08}\\ 

 \ganttbar{Construção da Amostra}{07}{08}\\ 
 \ganttbar{Construção das Técnicas}{07}{08}\\ 
 
\ganttgroup{Coleta dos Dados}{09}{11}\\ 
  
 \ganttbar{Lattes de Doctors formados 2002-2006}{09}{09}\\
 \ganttbar{Lattes de Doctors formados 2007-2011}{10}{10}\\ 
 \ganttbar{Lattes de Doctors formados 2012-2016}{11}{11}\\ 
 
\ganttgroup{Descrição}{12}{15}\\ 

 \ganttbar{Preparação e Tratamento dos Dados}{12}{14}\\ 
 \ganttbar{Análise dos Resultados}{15}{16}\\
 
\ganttgroup{Interpretação}{16}{21}\\ 

 \ganttbar{Análise Interpretativa}{16}{16}\\ 
 \ganttbar{Conclusão}{17}{17}\\
 \ganttbar{Escrita do Relatório Técnico}{18}{18}\\
 \ganttbar{Elaboração da Dissertação}{19}{21}\\ 
 
\ganttmilestone{Entrega da Dissertação}{22}{22}\\  
\ganttmilestone{Apresentação de Dissertação}{23}{24}\\ 

%%% Relações %%%%%%%%%%%%
\ganttlink{elem3}{elem6}
\ganttlink{elem6}{elem7}
\ganttlink{elem7}{elem8}
\ganttlink[link type=dr]{elem8}{elem10}
\ganttlink{elem10}{elem11}
\ganttlink{elem11}{elem14}
\ganttlink{elem14}{elem15}
\ganttlink{elem15}{elem16}
\ganttlink{elem16}{elem17}
\ganttlink{elem17}{elem18}

\ganttlink{elem2}{elem5}
\ganttlink{elem5}{elem9}
\ganttlink{elem9}{elem12}
\ganttlink{elem12}{elem17}
%%%%%%%%5%%%%%%%%%5%%%%%%%
\end{ganttchart}
} % PS
\end{figure}
\end{frame}
%%5%%%%%%%%%%%%%%%%%%%%%%%%%%%5%%%%%

\end{document}





